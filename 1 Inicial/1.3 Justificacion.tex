
\section{Justificación}

El aprendizaje eficiente del método \textbf{Fridrich/CFOP} depende menos de la explicación teórica y más de la \textbf{práctica deliberada, repetible y con retroalimentación inmediata}. 
La literatura en aprendizaje motor muestra que \textbf{aumentar el volumen de repeticiones de calidad}, con condiciones controladas y feedback oportuno, acelera la \textbf{automatización} de la destreza y reduce la \textbf{carga cognitiva} durante la ejecución \cite{haith2019shape,oppici2021deliberate,du2022relationship}. 
En la práctica cotidiana, sin embargo, el estudiante invierte tiempo y atención en \textbf{preparar manualmente} los casos (OLL, PLL, F2L), lo que disminuye el número de intentos útiles, introduce errores y \textbf{frustra} el proceso de dominio progresivo del algoritmo.

El \textbf{sistema mecatrónico propuesto} elimina ese cuello de botella al \textbf{preparar automáticamente} el estado objetivo aplicando el \textbf{inverso} del algoritmo humano y, además, al \textbf{verificar} dicho estado con visión por computadora. 
Esta combinación permite al alumno concentrarse en la \textbf{ejecución y memorización}, elevando el ritmo de práctica sin sacrificar precisión; al mismo tiempo, reduce la frustración asociada a fallos de preparación y a la incertidumbre sobre si el caso es el correcto \cite{khan2020opencv,appapogu2016rubik,hack2015rubik}. 
Desde la perspectiva de \textbf{HCI y educación mecatrónica}, integrar una \textbf{interfaz gráfica} con \textbf{feedback sensorial} (visión) mejora el compromiso, la motivación y la efectividad del aprendizaje en entornos de ingeniería, lo que respalda el valor didáctico del entrenador CFOP propuesto \cite{guerrero2023educational,aljuboori2022application}.

En términos de \textbf{aplicación educativa}, el sistema puede utilizarse en \textbf{laboratorios STEM} y clubes escolares como \textbf{plataforma abierta} para enseñar principios de mecánica (tolerancias, rigidez, backlash), electrónica de potencia, control secuencial y visión por computadora. 
Su arquitectura modular facilita prácticas y proyectos por etapas (del GUI al firmware), con \textbf{métricas objetivas} de aprendizaje: tiempo por intento, número de repeticiones, tasa de acierto, y evolución de la ejecución a través de sesiones \cite{guerrero2023educational,aljuboori2022application,budynas2011shigley,shigley2004handbook}.

Para la \textbf{comunidad de speedcubing}, la herramienta habilita \textbf{entrenamiento por casos} a alta cadencia (p. ej., ``OLL 21'', ``PLL T''), manteniendo la \textbf{consistencia mecánica} entre repeticiones---algo que los \textbf{robots resolutores} de estado del arte no priorizan, pues buscan minimizar movimientos globales y no respetar etapas humanas del CFOP \cite{kociemba1992two,rokicki2014diameter}. 
Con objetivos técnicos como \textbf{tiempo de preparación $\leq 5$ s}, \textbf{precisión $\pm 2^\circ$ por giro} y \textbf{$\geq 30$ repeticiones estables} por caso sin recalibración, el sistema apunta a un entorno de práctica \textbf{confiable, reproducible y medible}, alineado con los principios de ingeniería de diseño mecánico y control \cite{budynas2011shigley,shigley2004handbook}.

Finalmente, al ser \textbf{código abierto} y de \textbf{bajo costo relativo}, la solución favorece su \textbf{transferencia} y adopción en contextos de educación pública y clubes independientes, potenciando tanto la \textbf{equidad de acceso} como la \textbf{investigación aplicada} en didáctica de algoritmos, visión por computadora y control de actuadores \cite{guerrero2023educational,aljuboori2022application}. 
En conjunto, el impacto esperado es una \textbf{aceleración del aprendizaje}, una \textbf{reducción de la frustración} operativa y un \textbf{marco medible} para el entrenamiento sistemático de CFOP.
