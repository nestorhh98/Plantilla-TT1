\section{Definición del problema}

En el aprendizaje del método \textbf{Fridrich/CFOP} (Cross–F2L–OLL–PLL) para el cubo Rubik 3×3, el principal cuello de botella no es la explicación teórica de los algoritmos sino la \textbf{práctica deliberada y repetible de casos específicos.} Hoy, quien entrena debe \textbf{preparar manualmente} el cubo hasta obtener un caso (p. ej., un OLL o un PLL concretos), lo cual consume tiempo, introduce errores y \textbf{reduce el volumen efectivo de práctica} por sesión. La evidencia sobre práctica motora muestra que \textbf{incrementar repeticiones de calidad} acelera la ejecución, favorece la automatización del gesto y \textbf{disminuye la carga cognitiva}[1]-[3], pero esto depende críticamente de poder \textbf{repetir el mismo caso de forma confiable} (las condiciones de práctica importan).

Existen robots que \textbf{resuelven} el cubo extremo a extremo (visión del estado → cálculo de solución → ejecución), pero \textbf{no están orientados al entrenamiento por bloques} ni a \textbf{preparar y verificar un caso humano-estructurado} (p. ej., “OLL 21” o “PLL T”) de manera rápida y repetible; su objetivo es resolver en el menor número de movimientos, no respetar etapas CFOP ni facilitar repeticiones controladas [4]-[7].

Además, cuando se intenta incorporar \textbf{visión por computadora} para verificar el estado antes/después de los giros, surgen retos prácticos: \textbf{variación de iluminación}, \textbf{clasificación robusta de colores} en stickers, segmentación por contornos y umbrales en espacios HSV/Lab, así como consistencia de lectura entre ciclos [8]-[10]. Estos retos han sido reportados en implementaciones abiertas y preguntas técnicas frecuentes de la comunidad.

Por otra parte, desde la perspectiva de \textbf{educación mecatrónica y HCI}, está documentado que los sistemas didácticos con \textbf{interfaz gráfica + feedback sensorial} (p. ej., visión) mejoran la motivación y la efectividad del aprendizaje en prácticas de ingeniería, lo que sugiere valor pedagógico para un \textbf{entrenador CFOP} con verificación automática del estado [11]-[12].

\textbf{Problema a resolver}. No existe (al alcance del estudiante promedio) un \textbf{sistema mecatrónico abierto específico} para entrenamiento \textbf{CFOP por casos}, capaz de:

\textbf{preparar automáticamente} un caso desde cubo resuelto aplicando el \textbf{inverso} del algoritmo humano deseado,

\textbf{verificar} con visión que el estado generado es correcto, y

\textbf{permitir repeticiones rápidas y seguras} con precisión angular y consistencia mecánica suficientes.

\textbf{Consecuencia.} Sin esta herramienta, el estudiante \textbf{pierde tiempo} preparando el caso, \textbf{reduce el número de repeticiones útiles}, introduce errores al montar el estado y limita su progreso hacia la \textbf{automatización} y \textbf{disminución de carga cognitiva} propias del dominio experto [1]-[3].

\textbf{Reto de ingeniería}. Diseñar e integrar un \textbf{Sistema Mecatrónico de Entrenamiento CFOP} que combine:

\textbf{Mecánica} de sujeción y giro (tolerancias, backlash, par/rigidez) para 90°/180° repetibles;

\textbf{Electrónica} de potencia/actuación segura y estable;

\textbf{Informática/HCI} (GUI) con base de algoritmos y cálculo de inversos;

\textbf{Control} temporizado y sincronizado con visión;

\textbf{Visión por computadora} robusta a iluminación/colores;
y validar su \textbf{precisión angular, tiempo de preparación} por caso y \textbf{tasa de éxito} en ciclos repetidos. Referencias de robots resolutores y prácticas de diseño mecánico reportan la importancia de reducir juego mecánico, seleccionar materiales/rodamientos adecuados y proteger la transmisión del par [13]-[14].